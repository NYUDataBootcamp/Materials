%%%%%%%%%%%%%%%%%%%%%% Stock Preamble %%%%%%%%%%%%%%%%%%%%%%%%%%%%%%%%%%

\documentclass[12pt,pdftex,twoside,letterpaper]{exam}
\usepackage{amsmath,amssymb, amsthm}
\usepackage{graphicx}
\usepackage{color}
\usepackage{comment}
\usepackage{layout}
\usepackage{booktabs}
\usepackage[flushleft]{threeparttable}
\usepackage{caption}
\usepackage{setspace}
\usepackage{float}
\usepackage{needspace}
\usepackage[colorlinks=true,linkcolor=blue,citecolor=black,urlcolor=blue,bookmarks=false,pdfstartview={FitV}]{hyperref}

%%%%%%%%%%%%%%%%%%%%%%Margins%%%%%%%%%%%%%%%%%%%%%%%%%%%%%%%%%%%%%%%%%%%%%%%
\usepackage[margin=1.0in]{geometry}
\setlength{\parindent}{0in}
\setlength{\parskip}{.09in}
\raggedbottom

%%%%%%%%%%%%%%%%%%%%Tighten up the lists%%%%%%%%%%%%%%%%%%%%%%%%%%%%%%%%%%
\let\OLDdescription\description
\renewcommand\description{\OLDdescription\setlength{\itemsep}{-2mm}}

%%%%%%%%%%%%%%%%%%%%%%%%%Exam class formating%%%%%%%%%%%%%%%%%%%%%%%%%%%%%%%%%%%%%%%%
%\printanswers
\renewcommand{\partlabel}{\thepartno.}
\renewcommand{\questionshook}{\setlength{\itemsep}{0.2in}}
\renewcommand{\partshook}{\setlength{\leftmargin}{0.2in}}
\renewcommand\familydefault{\sfdefault}

%%%%%%%%%%%%%%%%%%%%%%%%%%%%%%%%%Let Backus fight the good fight, I'm going with LN%%%%%%%%%%%%%%%%%%%
\renewcommand{\log}{\ln} 

%%%%%%%%%%%%%%%%%%%%%%Headers and Footers%%%%%%%%%%%%%%%%%%%%%%%%%%%%%%%%%%
\pagestyle{headandfoot}
\runningheadrule
\firstpageheadrule
\firstpageheader{\includegraphics[width=0.25\textwidth]{../Figures/stern_black1.pdf}}{}{Data Bootcamp: Three Project Ideas}
\runningheader{}{}{}
\runningfooter{}{}{}

\begin{document}
\bigskip
\centerline{\Large \bf Topic Outline:  Data + Python}
\medskip
\centerline{Revised: \today}


\section*{Materials}

\begin{itemize}
\item  Today's handouts:  Syllabus, due dates, this outline, red/green stickers
\item  All posted on website (except the stickers).
\end{itemize}

\section*{About the course}

\begin{itemize}
\item Data + Python = Magic!
\begin{itemize}
\item Arthur C. Clarke, Jessica, Tim
\end{itemize}

\item What?
\begin{itemize}
\item ... are you doing here?
\item Skills are nice, coding is literacy for the modern age
\item Something to show potential employers
\end{itemize}

\item Why?
\begin{itemize}
\item Why data?
\item Why code?
\item Why Python?
% general purpose language, great data tools, open source and free, community
\item Why bootcamp?
\item Why you?
\end{itemize}

\item Things we believe
\begin{itemize}
\item Anyone can do this.  Target audience is {programming newbies --- with courage}.
\item It's ok to be lost.  We've all been there, it's not permanent.
%\item We're here to help.  But you need to tell us when you're lost.
\item This is fun.  Really.
%It's an amazing feeling to be able to do cool things in minutes.
\end{itemize}


\item Rules to live by
\begin{itemize}
\item Don't panic.  It will seem overwhelming at first, but stick with it and you'll be fine.
\item One step at a time.  Don't rush this.  In six weeks you'll know a lot.
\item Learn by doing.  Same directions as Carnegie Hall, no shortcuts.
%\item Let your nerd flag fly.  Learn to love xkcd.
\item Ask for help.  Don't be a hero, let us know if you could use some help.
\end{itemize}

\needspace{2\baselineskip}
\item Course materials
\begin{itemize}
\item Required:  practice, exam, project
\item Google ``nyu data bootcamp''
\item Website, topic list \& links  (thanks, Spencer):\\ \url{https://nyu.data-bootcamp.com/undergrad_outline/}  (bookmark me!)
\item Book: \\ \url{https://www.gitbook.com/book/nyudatabootcamp/data-bootcamp/details}
\item Discussion group: \\ \url{https://groups.google.com/d/forum/databootcamp_fall2017_undergrad}
\item Data page: \\
\url{https://nyu.data-bootcamp.com/data/}
\item GitHub repository:  \url{https://github.com/NYUDataBootcamp/Materials}
\end{itemize}

\item You
\begin{itemize}
\item Come to class
\item After class:  {\bf write} and {\bf read}
\item Practice
\item Have fun
\end{itemize}
\end{itemize}

\section*{Setting up your Computer}
\begin{itemize}
\item Create \verb|Data_Bootcamp| directory/folder on your computer. This is a place for you to save stuff to, work from, etc.
\item My style: To keep things simple I do the following
\begin{itemize}
\item Find your main hard drive, on PCs typically the ``C'' drive
\item In the C drive, create a folder called \verb|Data_Bootcamp|
\item Now if you every have to call this file you know it is \verb|c:\Data_Bootcamp|
\end{itemize}
\end{itemize}

\section*{\href{https://atom.io/}{Atom}}
\begin{itemize}
\item A versatile, text editor. Must have tool for serious work.
\item We will use it to open files, look at them, learn some markdown.
\item Lets install it
\begin{itemize}
\item Put red sticker on your laptop
\item Google ``atom download'' or borrow a USB drive
\item Download or copy installer to your computer
\item Run installer
\item Start Atom
\item Replace red sticker with green when Atom opens
\end{itemize}
\item Go and open new file
\item type my first file, then comma, then birth year, then comma, high school graduation year, then comma, college graduation year (or expected graduation year). For me it is like this\\
    \\
{\tt my first file, 1978, 1996, 2001} 
\item Save as: \verb|my-first-file.csv| in your \verb|Data_Bootcamp| file.
\item \textbf{Pro-tip: naming convention and file type is important in this course}
\end{itemize}


\section*{\href{https://github.com}{GitHub}}
\begin{itemize}
\item What I will use it for\ldots
\begin{itemize}
\item Source of ALL course materials
\item Place for you to grab materials on the fly. Save files by cut and paste, clever save as, or "Raw" (ask about this)
\end{itemize}
\item You need to create an account and email me your username
\item What you will use it for\ldots
\begin{itemize}
\item Post your homework and projects; I will ``pull'' them from there.
\item Long run: Think of this like an artists portfolio. Here you can post your code and projects and show potential employers, family, friends what you have done.
\end{itemize}
\item Now lets use it
\begin{itemize}
\item Create a new repository and name it {\tt my-first-repository},
\item place the \verb|my-first-file.csv| file you created in it,
\item Great job! Next class, I'm going to try and ``pull'' the file and then report some statistics about the class to you.
\end{itemize}
\end{itemize}


\section*{Anaconda}

\begin{itemize}
\item Install the Anaconda distribution
\begin{itemize}
\item Put red sticker on your laptop
\item Distribution?
\item Google ``anaconda download'' or borrow a USB drive
\item Download or copy installer to your computer --- {\bf Python 3.6!}
\item Run installer
\item Start Launcher (use search box)
\item Replace red sticker with green when Launcher opens
\end{itemize}

\item Environments
\begin{itemize}
\item Environments?  (Analogy:  Word is an environment for creating Word docs.)
\item Spyder:  classic coding environment with editor and output windows
\item Jupyter:  environment for creating IPython notebooks, which combine code with text and output
\end{itemize}
\end{itemize}



\section*{Run test program -- twice}

\begin{itemize}
\item Test program code:

\vspace{-0.1in}
\begin{verbatim}
"""
Test program for Data Bootcamp course @ NYU Stern
"""
import sys

print('Welcome to Data Bootcamp!')
print('Python version:')
print(sys.version)
\end{verbatim}

\needspace{2\baselineskip}
\item Run test program in Spyder
\begin{itemize}
\item Put red sticker on your laptop
\item From Launcher, launch Spyder (labelled ``spyder-app'')
\item Look around (editor, IPython console, Object inspector)
\item Enter test program in editor (on the left)
\item Save in \verb|Data_Bootcamp| directory as \verb|bootcamp_test.py|
(File, Save as, look for folder)
\item Run program (click on large green triangle)
\item Look for correct output (last line should be {\tt 3.6.x etc})
\item Switch to green sticker if it works
\end{itemize}

% Not going to talk about Jupyter now...
%\item Run test program in Jupyter
%\begin{itemize}
%\item Put red sticker on your laptop
%\item From Launcher, launch Jupyter (labelled ``ipython-notebook'')
%\item Navigate to \verb|Data_Bootcamp| directory
%\item Open a new IPython notebook (New, Python 3)
%\item Change name from {\tt Untitled} to \verb|bootcamp_test|
%\item Look around (toolbar, menubar, code cells)
%\item Enter test program in code cell
%\item Run program (Cell, Run All)
%\item Look for correct output (last line should be {\tt 3.5.x etc})
%\item Switch to green sticker if it works
%\end{itemize}

\item Spyder startup summary
\begin{itemize}
\item Open by typing Launcher in search box (spotlight on Macs), then choose spyder-app.
\item Or just type Spyder in search box
\end{itemize}
\end{itemize}

\section*{Practice and review}


Put red sticker on your laptop, replace with green when you're done.
Discuss with your neighbor.
Raise your hand if you could use some help.

\begin{enumerate}

\item Fill in the blanks in this table:
%relating ``environments'' to the files they are related to:

\begin{center}
\begin{tabular}{cc}
\toprule
Environment & File or Object \\
\midrule
MS Word  & Word document  \\
 & Excel file     \\
iTunes & \\
%Typewriter & \\
Atom & \\
Spyder   &                \\
\bottomrule
\end{tabular}
\end{center}

\item Open Atom and a new file. 
\begin{itemize}
\item In a similar fashion as above, create a {\tt .csv } file with the following information separated by commas
\begin{itemize}
\item Expected major (or concentration)
\item Scale of 1-5 (5 being a lot, 1 being none) experience with programming
\item preferred form of social media
\item career path (e.g. Finance, Consulting, Technology, Fashion, public policy, rock star)
\end{itemize}
\item Name as \verb|my-second-file.csv| and post on GitHub in {\tt my-first-repository}
\end{itemize}
    

\item Run the \verb|Maddison_data_input.py| Python code example.
\begin{itemize}
\item Go to the \verb|Data_Bootcamp| GitHub repository (link above).
\item Navigate to the {\tt Code} directory and {\tt Lab} subdirectory.
\item Get \verb|Maddison_data_input.py|
\begin{itemize}
\item Cut and paste into blank file
\item Or:  Save file in \verb|Data_Bootcamp| directory (ask how)
\end{itemize}
\item Open file in Spyder (File, Open).
\item Run it by clicking on large green triangle.
\item What do you see?
\end{itemize}

\item {\it Only if you have time.\/} Try this program: \verb|OECD_health_indicators.py|.
What do you see?  What questions does it raise?
(There are other files in the same directory, but some of them don't work yet.)

\end{enumerate}

\section*{After class}

\begin{itemize}
\item Required
\begin{itemize}
\item Read Syllabus and Project Guide.
\item Mark Due Dates on your calendar.
\item Skim chapters 1-3 of the book.
\end{itemize}
\item Recommended
\begin{itemize}
\item If you haven't already:  join the discussion group.
\item Explore the website.  Make sure you can find the book, due dates,
topic outlines, assignments, and data sources.
\item Post a link to an interesting graph on the discussion group.
\item Look through the IPython notebook \verb|bootcamp_examples.ipynb|
in the {\tt Code/IPython} directory of the GitHub repo.
What graphs interest you?  What data?
Do they suggest anything else you might explore?
\end{itemize}
\end{itemize}

\end{document}

